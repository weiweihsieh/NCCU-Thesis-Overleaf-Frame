\chapter{結論與建議}
\label{c:experiment}

\section{註腳範例}
%使用\footnote{}可以插入註腳

狗狗看不出顏色,所以看到警犬跟警察其實分不出來。狗狗看不出顏色,所以看到警犬跟警察其實分不出來。狗狗看不出顏色,所以看到警犬跟警察其實分不出來。狗狗看不出顏色,所以看到警犬跟警察其實分不出來。狗狗看不出顏色,所以看到警犬跟警察其實分不出來。狗狗看不出顏色,所以看到警犬跟警察其實分不出來。狗狗看不出顏色,所以看到警犬跟警察其實分不出來。狗狗看不出顏色,所以看到警犬跟警察其實分不出來。狗狗看不出顏色,所以看到警犬跟警察其實分不出來。狗狗看不出顏色,所以看到警犬跟警察其實分不出來。狗狗看不出顏色,所以看到警犬跟警察其實分不出來。狗狗看不出顏色,所以看到警犬跟警察其實分不出來。狗狗看不出顏色,所以看到警犬跟警察其實分不出來。狗狗看不出顏色,所以看到警犬跟警察其實分不出來。狗狗看不出顏色,所以看到警犬跟警察其實分不出來。狗狗看不出顏色,所以看到警犬跟警察其實分不出來。狗狗看不出顏色,所以看到警犬跟警察其實分不出來。\footnote{引注自:Discovery}


\section{參考文獻引用方法}
%首先需要獲取所需引用的參考文獻的APA檔,可以自行google或參考GitHub repo的README。

%先將所參考文獻的APA檔寫至thesis.bib中,即可在正文中使用\cite{}來飲用參考文獻。

根據\cite{slabbert1999early}的研究中所提出的觀點,狗可能真的可以分辨出警犬跟一般犬(並沒有!)。